\section{Future Work}
\label{sec:fut}

%Introduction
It is advised to do several researches and to take measures prior to a feasibility study and implementation of a Job Recommender System (JRS).
The most important ones are described briefly in this section.

%prerequisites
Before a feasibility study of a JRS can be done measures and other research is advised.
Measures can be taken to improve the registration of job matches outcomes.
Measures can be taken to reduce the bias, error and irregularities in the data processing.
Here standardized procedures can be researched and implemented.
This kind of preliminary research would be more in the field of Business Economics, Lean Six Sigma, and/or Operational Excellence than in the field of Data Science (alone).
In anticipation of that an in depth social research is advised to determine which user attributes are the most important for finding a job.
Finally, it can be researched if jobs can be grouped in such a way that similar jobs have similar hiring criteria.

%follow-up studies
When the quantity and quality of data is improved a new JRS feasibility study can be done.
It is advised to split up this study  in two phases (or even in two  projects) with multiple Data Science researchers because there are many facets to a JRS. 
Moreover it is very well possible that when the new data are available there might be different JRS solutions that can be investigated.
Therefore the first phase should emphasize on researching the most promising methods for a JRS based on the new data.
.The second phase is to conduct a user study to validate the most promising methods for the JRS in a real life setting.
It can be imagined that such an approach could be a collaboration between a Data Scientist and a Social and/or Economic researcher.  

%conclusion
Building a JRS is quite comprehensive, to do it properly it requires multiple extensive research phases or projects preceding that.