\section{Future Work}
\label{sec:fut}

%Introduction
The results in section 5 showed a number of challenges for the implementation of a Job Recommender System (JRS).
% However given the results it is impossible to recommend general research extensions in the field of JRSs.
% \mynote[author=Harrie]{Je hoeft niet eerst te focussen op wat je niet kan doen, begin anders direct met future work voor Amsterdam.}
The future work is limited to the research that can be done based on this study for a JRS for welfare beneficiaries within the municipality of Amsterdam.
It is advised to do several researches and to take measures prior to a new feasibility study of a JRS.
The most important ones are described briefly in this section.

%prerequisites
Measures can be taken to improve the registration of job matches outcomes.
Alternatives for the documentation of feedback by the employers in GIP can be researched, such as using implicit feedback.
The termination of welfare benefits payouts can for example be interpreted as a signal that a job match was successful. 
Measures can be taken to reduce the bias and noise in the data processing.
Here standardized procedures and user profiles can be researched and implemented.
This kind of preliminary research would be more in the field of Business Economics, Lean Six Sigma and/or Operational Excellence than in the field of Data Science (alone).
In anticipation of that an in depth social research is advised to determine which user attributes are the most important for finding a job.
As well, it can be researched if jobs can be grouped in such a way that similar jobs have similar hiring criteria.

%follow-up studies
When the quantity and quality of data is improved a new JRS feasibility study can be done.
It is advised to split this study into two phases (or even into two  projects) with multiple Data Science researchers because there are many facets to a JRS. 
Moreover it is very well possible that when the new data is available there might be different JRS solutions that can be investigated.
Therefore the first phase should emphasize on researching the most promising methods for a JRS based on the new data.
The second phase is to conduct a user study to validate the most promising methods for the JRS in a real life setting.
It can be imagined that such an approach could be a collaboration between a Data Scientist and a Social and/or Economic researcher.  

%conclusion
Building a JRS is quite comprehensive, to do it properly requires multiple extensive research phases or projects preceding that.
