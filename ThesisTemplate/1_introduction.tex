\section{Introduction}
\label{sec:intro}

% \mynote[author=Harrie]{Voorbeeld van commentaar vanuit Harrie.}

%body
% \todo{Motivatie voor job recommendation}
% \todo{Motivatie voor jouw werk (toevoeging/contributie)}
% \todo{uitlijning van wat er gaat komen}
% \todo{RQ}
% \todo{specificatie van jouw project met de gemeente Amsterdam}


%Motivatie voor job recommendations
The digitalization of our society has made information more easily accessible than ever before.
When, for example, buying a product or choosing a movie to watch people are presented with myriad of choices. 
However more choices do not always equal to better choices, this paradox of choice especially occurs when people are overloaded with information \cite{schwartz2004paradox}. 
Recommender Systems (RS) are software tools and techniques designed to aid people in making choices by selecting and presenting items to them that could be of their interest.

RSs are applied in many different domains \cite{aggarwal2016recommender}, famous examples of companies using RSs are Amazon (products and services), Netflix (movies), Spotify (music), Facebook (content) and LinkedIn (jobs).
For matching job seekers with job vacancies RSs are a useful tool to find the best job openings, or visa versa for recruiters to find the best candidates for a job.
The job recommendation problem is essentially different from other traditional recommendation problems such recommending products or movies to users.
The main difference is that a same product or movie can potentially be recommended to thousands of consumers, while a job vacancy is usually posted with the goal to hire one or only a couple of employees.

%Motivatie voor jouw werk (toevoeging/contributie)
Today the application of Job Recommender Systems (JRS) is mostly restricted to online platforms. 
LinkedIn for example claims that its JRS is crucial in helping to achieve its company goals \cite{kenthapadi2017personalized}.
While online job boards and professional social networks such as Indeed and Linkedin can aid most people in finding a job, they are not always suitable for people at the lower end of the labor market.
A group of people who are often positioned at the lower end of the labor market are welfare beneficiaries. 
The composition of the group of welfare beneficiaries is characterized by a strong representation of immigrants, older people, people with low education, and people who suffer from mental or physical disabilities \cite{dodeweerd}.
Due to the special needs and characteristics of welfare beneficiaries a different and novel application of a RS is required. 
The goal of this type of JRS is, by taking into account the special needs and characteristics of welfare beneficiaries, to make better and more lasting matches between them and job openings.
Such a system would operate at the intersection of RSs, information retrieval, machine learning and statistical optimization. 

The research presented in this paper will focus on the job recommendation problem and its related machine learning challenges for the group of welfare beneficiaries in the city of Amsterdam. 

%Main Research Question
The paper aims to answer the following Research Question:
\begin{enumerate}
    \item \em Can a Content-Based Recommender System based on matching job openings to welfare beneficiaries be comparable to human customer managers? \label{rq:mrq}
\end{enumerate}{}

%Sub Research Questions
The subquestions we answer are:
\begin{enumerate}\addtocounter{enumi}{1}
    \item Can clustering deal with label sparsity in job matching data? \label{rq:cold}
    \item Are there significant differences between models in predictive performance? \label{rq:model}
    \item Does a model trained for prediction of job matches translate to a good ranking-based recommender? \label{rq:ranking}
    \item Does feature selection and addressing class imbalance have a significant effect on the predictive performance? \label{rq:strategies}
\end{enumerate}

%uitlijning van wat er gaat komen
The paper is structured as follows. First, related previous studies are discussed (section \ref{sec:rel}). Next, the methodology and machine learning models are elaborated (section \ref{sec:meth}). This is followed by a description of the experimental setup (section \ref{sec:setup}). Hereafter the results are presented (section \ref{sec:rslts}). Then the outcomes and limitations of the research are discussed (section \ref{sec:disc}). After that the conclusions are drawn based on the findings (section \ref{sec:concl}). Finally, the future extensions of the research are explored (section \ref{sec:fut}).

%specificatie van jouw project met de gemeente Amsterdam
As part of the project “Optimalisatie Proces Begeleiden naar Werk en Participatie” the municipality of Amsterdam has supplied the data for this research. 
The aim of the this project is to reduce the number of welfare beneficiaries by getting them employed.
Creativity in applying novel technologies is conceived by the municipality to play an important role in achieving the objectives.
The learnings described in this paper can provide the city of Amsterdam with guidelines for the implementation of an intelligent and data driven system to match welfare beneficiaries with job opportunities. 



