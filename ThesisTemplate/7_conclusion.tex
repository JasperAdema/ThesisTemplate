\section{Conclusion}
\label{sec:concl}

%introduction
A Job Recommender System (JRS) is a complex system with many facets, implementing one is not a straightforward process.
In order to investigate the possibilities for a JRS for welfare beneficiaries we tested various machine learning techniques on data provided by the municipality of Amsterdam.
% \mynote[author=Harrie]{Zeg kort (1~2 zinnen) wat je hier gedaan hebt, iets als: we hebben verschillende methodes op de data van de gemeente amsterdam toegepast.}
In the Results section (\ref{sec:rslts}) it is shown that the methods that were tested during this research did not result in a working JRS.
In the section Discussion (\ref{sec:disc}) it is argued that this outcome can be attributed to the quantity and quality of the data, rather than to the machine learning models that were applied.
It is believed that a JRS is still possible, however not with the data that was available for this research project.

%Conclusion
In conclusion it is unlikely that a JRS based on the tested methods in combination with the available data can be comparable to human customer managers.
However, when in the future the data quantity and quality is improved a JRS might become feasible.