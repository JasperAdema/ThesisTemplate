\section{Methodology}
\label{sec:meth}

%Introduction
In this section the methods are tested during the research are elaborated. First the job recommendation setting is outlined (\ref{ssec:jrs}). Then the clustering methods are presented (\ref{ssec:clus}). This is followed by the content-based learning methods (\ref{ssec:cblup}). In the last part practical solutions to improve the recommendation results are covered (\ref{ssec:psirr}). 

%What is the recommendation setting?
\subsection{The Job Recommendation Setting}
\label{ssec:jrs}
The research setting governs for an important part which methods can be applied.
The dataset is composed of users features, job features and labels, and is a mixture of numerical, categorical, and unstructured (free text) data.
The users features are for example personal information such as age, education, housing, marital status, but also biographical information as personal traits and limitations. 
Examples of job features are the job title, job description, job sector and hiring criteria. 
The labels are the job matching outcomes, is a person hired for a job or not . 
The positive labels (a person is hired) are overrepresented in dataset due to the tendency to document only the most promising job matching attempts. 
Finally, the dataset exhibits sparsity, with just one or a few job matches per user.

%Clustering
\subsection{Clustering}
\label{ssec:clus}
The dataset contains only one or a couple of job matches per user, a method to mitigate this sparsity is to cluster the users. 
If a cluster of similar users is greater than one there are automatically multiple job matches linked to it. 
Similarly the jobs can be clustered, with the difference that now multiple users are connected to a cluster of jobs. 
Three approaches to clustering are explored: 1) the clustering of user and job features, 2) the clustering of job descriptions, and 3) the clustering of job titles. 

\noindent
\textbf{The Clustering of User and Job Features} \\
The dataset contains a mixture of data types and most clustering algorithms can handle only one data type and therefore two methods to cluster the user and job features are tested.
The first method is to use the K-prototypes algorithm \cite{huang1997clustering} for clustering which can handle both numerical and categorical data. 
The second method is to bin all numerical data and to treat it as categorical data and apply the K-modes clustering algorithm \cite{huang1997clustering, huang1998extensions} which can only deal with categorical data. 

\noindent
\textbf{The Clustering of Job Descriptions} \\
The job descriptions are unstructured data (free text) and are first processed with a Natural Language Processing (NLP) technique called Term Frequency–Inverse Document Frequency (TF-IDF) \cite{ramos2003using}. 
This technique returns a vector space of numerical data (the TF-IDF scores) on which the K-means clustering algorithm \cite{hartigan1979algorithm} can be applied. 

\noindent
\textbf{The Clustering of Job Titles} \\
The final clustering method involves extracting the n-grams of the characters in the job title and putting them into a vector space.
This vector space of n-gram frequencies is then clustered using hierarchical clustering \cite{rokach2005clustering}. 
The advantage of hierarchical clustering is that the number of clusters do not need to be specified. 

%CB models
\subsection{Content-Based Learning of User Profiles} 
\label{ssec:cblup}
The labels in the dataset can be leveraged for supervised machine learning algorithms. 
A content-based model can be constructed to predict the user matchability with jobs. 
For the learning of the profiles the user and jobs features are concatenated into a single vector per available label. 
The learning of these profiles is closely related to the regression and classification problem, and therefore the same methods can be applied. 

There is a myriad of regression and classification algorithms which all can be used for recommendation, and to narrow these down the following six basic models were selected for this research: 1) Nearest Neighbors, 2) K-Nearest Neighbors Classification, 3) RandomForest Classification, 4) Logistic Regression, 5) Linear Support Vector Classification, and 6) Multi-Layer Perceptron Classifier (better known as a Neural Network).
All mentioned methods can return a predicted label based on a probability estimate.
The motivation for selecting each of these models is explained in brief.

\textbf{\textit{Nearest Neighbors.}}
Other state-of-the-art algorithms commonly deliver a better performance than Nearest Neighbors \cite{koren2008factorization, takacs2007major}, however it is the least complex of all proposed classification algorithms and straightforward to implement while nevertheless providing good results.
Ning et al. (2015) summarize the advantages of Nearest Neighbors as follows: a) simplicity, b) justifiability, c) efficiency, and d) stability \cite{ning2015comprehensive}.
Because of these characteristics the outcomes of Nearest Neighbors classification will serve as a baseline to which the other proposed algorithms will be compared.
\textbf{\textit{K-Nearest Neighbors}} \cite{altman1992introduction} is an extension of the Nearest Neighbors algorithm. 
In fact Nearest Neighbors is the equivalent of K-Nearest Neighbors (KNN) with a k=1. 
Thus the difference between the two is that with KNN the desired number of neighbors can be specified.
The arguments for selecting this neighborhood method are the same as for Nearest Neighbors.

\textbf{\textit{RandomForest}} \cite{breiman2001random} is an ensemble learning method that can be used to build a rule-based classifier, when classification rules are extracted from the model.
The advantage of the RandomForest classification algorithm it that is robust to the inclusion of irrelevant features.

\textbf{\textit{Logistic Regression}} \cite{hosmer2013applied} assumes that the labels (job matching outcomes) can be modeled as a linear function of the features.
The prediction problem can be regarded as binary (0:negative label, 1:positive label), and therefore Binary Logistic Regression is applied.

According to Burges (1998) \textbf{\textit{Support Vector Machines}} (SVM) are an often applied method in case of binary labels \cite{burges1998tutorial}.
An SVM is comparable with Logistic Regression, the main difference is that the loss is quantified as a hinge-loss rather than a logit function. 
For this research a member of the SVM family called Linear Support Vector Classification is tested.

While Logistic Regression and SVMs excel in learning linear patterns in the data, a \textbf{\textit{Neural Network}} \cite{bishop1995neural} outperforms these when it comes to learning arbitrarily complex non-linear models.
If there are non-linear relations between features present in the dataset the Neural Network can learn those.
However, the drawbacks are that a Neural Network is sensitive to noise in the data and that it is prone to overfitting when the dataset is small. 

%How can we recommend?
\noindent
\textbf{Recommending Jobs to Users} \\
The outcomes of the previous mentioned models can be two types of recommendation: 1) the prediction of the utility of a certain job to a user, or 2) a list of the \textit{n}-top useful jobs for a user.
The first method is feeding a trained model a concatenated user and a job feature vector and predict the label.
The second method is to let the model predict the label probabilities of all possible users and jobs combinations. 
The outcomes are transformed to a users $\times$ jobs matrix in which each row represents a unique user.
Hereafter the probabilities in each row are ranked descending, and from that the \textit{n}-top ranks (jobs) are returned. 

%How to improve the results?
\subsection{Practical Solutions to Improve Recommendation Results}
\label{ssec:psirr}
In this part two strategies to deal with the main challenges of the data and to improve the recommendation results are described.

The first approach emphasizes on reducing the effects of the unequal distribution of positive and negative labels, this is also known as \textit{class imbalance}. 
The methods to mitigate class imbalance that are tried out are undersampling and oversampling. 
In case of undersampling observations of the overrepresented class will be randomly deleted until the classes are balanced. 
With oversampling artificial observations are created for the underrepresented class until the classes are balanced. 
For the oversampling a variation of the Synthetic Minority Over-Sampling Technique (SMOTE) \cite{chawla2002smote} suitable for both numerical and categorical data is applied. 

Focus of the second strategy is noise reduction by removing the features that contribute the least to the prediction of the labels.
Using only the important features reduces overfitting, improves accuracy and reduces training time. 
The three methods to determine feature importance that are tested are: 1) Chi-square test for feature selection, 2) Backwards Feature Elimination (BFE), and 3) Recursive Feature Elimination with Cross Validation (RFECV) \cite{guyon2002gene}. 
The Chi-square test is mostly applied for categorical data only, while the other methods can deal with mixed data types.
